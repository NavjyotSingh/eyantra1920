\documentclass[12pt]{article}
\usepackage{amsmath}
\usepackage{latexsym}
\usepackage{amsfonts}
\usepackage[normalem]{ulem}
\usepackage{soul}
\usepackage{array}
\usepackage{amssymb}
\usepackage{extarrows}
\usepackage{graphicx}
\usepackage[backend=biber,
style=numeric,
sorting=none,
isbn=false,
doi=false,
url=false,
]{biblatex}\addbibresource{bibliography.bib}

\usepackage{subfig}
\usepackage{wrapfig}
\usepackage{wasysym}
\usepackage{enumitem}
\usepackage{adjustbox}
\usepackage{ragged2e}
\usepackage[svgnames,table]{xcolor}
\usepackage{tikz}
\usepackage{longtable}
\usepackage{changepage}
\usepackage{setspace}
\usepackage{hhline}
\usepackage{multicol}
\usepackage{tabto}
\usepackage{float}
\usepackage{multirow}
\usepackage{makecell}
\usepackage{fancyhdr}
\usepackage[toc,page]{appendix}
\usepackage[hidelinks]{hyperref}
\usetikzlibrary{shapes.symbols,shapes.geometric,shadows,arrows.meta}
\tikzset{>={Latex[width=1.5mm,length=2mm]}}
\usepackage{flowchart}\usepackage[paperheight=11.69in,paperwidth=8.27in,left=1.0in,right=1.0in,top=1.0in,bottom=1.0in,headheight=1in]{geometry}
\usepackage[utf8]{inputenc}
\usepackage[T1]{fontenc}
\TabPositions{0.5in,1.0in,1.5in,2.0in,2.5in,3.0in,3.5in,4.0in,4.5in,5.0in,5.5in,6.0in,}

\urlstyle{same}

\setcounter{tocdepth}{5}
\setcounter{secnumdepth}{5}


\setlistdepth{9}
\renewlist{enumerate}{enumerate}{9}
		\setlist[enumerate,1]{label=\arabic*)}
		\setlist[enumerate,2]{label=\alph*)}
		\setlist[enumerate,3]{label=(\roman*)}
		\setlist[enumerate,4]{label=(\arabic*)}
		\setlist[enumerate,5]{label=(\Alph*)}
		\setlist[enumerate,6]{label=(\Roman*)}
		\setlist[enumerate,7]{label=\arabic*}
		\setlist[enumerate,8]{label=\alph*}
		\setlist[enumerate,9]{label=\roman*}

\renewlist{itemize}{itemize}{3}
		\setlist[itemize]{label=$\cdot$}
		\setlist[itemize,1]{label=\textbullet}
		\setlist[itemize,2]{label=$\circ$}


\pagestyle{fancy}
\fancyhf{}
\chead{ 
\vspace{\baselineskip}
}
\renewcommand{\headrulewidth}{0pt}
\setlength{\topsep}{0pt}\setlength{\parindent}{0pt}


\renewcommand{\arraystretch}{1.3}


%%%%%%%%%%%%%%%%%%%% Document code starts here %%%%%%%%%%%%%%%%%%%%

\begin{document}
\tab  \tab {\fontsize{18pt}{21.6pt}\selectfont \textbf{e-Yantra Ideas Competition 2019-20}\par}\par

\setlength{\parskip}{8.04pt}
{\fontsize{14pt}{16.8pt}\selectfont \textbf{\uline{Project Name:\tab }}\par}\par

AIDE: AI $\&$  IoT-enabled home automation for Disabled $\&$  Elderly\par


\vspace{\baselineskip}
{\fontsize{14pt}{16.8pt}\selectfont \textbf{\uline{Introduction/Motivation:}}\par}\par

\begin{justify}
With the rapid growth of technology, it has become important to explore different parts of automation in our day to day lives. Home automation means controlling electronic and electrical devices along with other manually operated things such as doors, windows etc to provide maximum efficiency [1]. 
\end{justify}\par

\begin{justify}
According to United Nations statistics, the disabled account for 10 percent of the population in the world. Census 2011(2016 updated) has revealed that over 22 million people in India as suffering some kind of disability. This is equivalent to 2.21$\%$  of the population [2]. Taking proper care of elderly, impaired or disabled people living in the comfort of their homes and being assisted by an automation system that helps them with their daily tasks like to open/close doors, turn on/off electrical devices and monitors their vital health parameters is now possible.
\end{justify}\par

\begin{justify}
While there are plenty of home automation systems that help to control the appliances by voice, they do not uniformly cater to the needs of all disabled people such as the dumb, the elderly who cannot speak loud enough, etc. This project focuses on devising a home automation system that serves the needs of a very wide range of disabilities. 
\end{justify}\par


\vspace{\baselineskip}
{\fontsize{14pt}{16.8pt}\selectfont \textbf{\uline{Market Research / Literature Survey:}}\par}\par

\begin{justify}
Many home automation systems have been developed in recent times. Some of them have been specially modified to cater to the needs of disabled and the elderly. They use hardware such as Arduino or Raspberry Pi as the central hub, controlled by voice commands and usually can also be operated from a mobile application [3].
\end{justify}\par

\begin{justify}
There are special features to bring about ease of living and care for people who are unable to fend for themselves. Most systems proposed a design which consists sensors such as light to regulate lighting, gas sensors to detect gas leaks [4], ventilation controls, temperature and humidity sensors for regulating thermostats [5][6] etc. 
\end{justify}\par

\begin{justify}
CarePredict’s Tempo is a wrist-worn bracelet with built-in microphone, speaker and sensors that detect activities of daily living (ADL). With the aid of sophisticated AI algorithms, Tempo can sense ADLs such as eating, bathing, toilet use, walking, sitting, sleeping etc [7].
\end{justify}\par


\vspace{\baselineskip}
{\fontsize{14pt}{16.8pt}\selectfont \textbf{\uline{Hardware requirements:}}\par}\par

\begin{enumerate}[label*={\fontsize{12pt}{12pt}\selectfont \arabic*.}]
	\item Arduino Uno, Raspberry Pi to work as central hub.\par

	\item Flex Sensors and EMG Sensors to sense gestures.\par

	\item NodeMCU WiFi Module for connection between appliances and hub.\par

	\item Accelerometer MPU6050 to detect movements and for recognising activities.\par

	\item M212 Pulse Sensor\par

	\item Infineon DPS310 Pressure Sensor\par

	\item DHT11 Temperature Sensor and MQ135 Gas Sensor\par

	\item Relays, MOSFETs\par

	\item Servo motors, stepper motors
\end{enumerate}\par


\vspace{\baselineskip}
{\fontsize{14pt}{16.8pt}\selectfont \textbf{\uline{Software requirements:}}\par}\par

\begin{enumerate}[label*={\fontsize{12pt}{12pt}\selectfont \arabic*.}]
	\item Arduino IDE\par

	\item Python 3.7+ \par

	\item Jupyter Notebook and PyCharm IDEs for AI training.\par

	\item MySQL and NoSQL databases for storing user data and the result of analysis on the acquired dataset.\par

	\item Android Studio for App Development.
\end{enumerate}\par


\vspace{\baselineskip}
{\fontsize{14pt}{16.8pt}\selectfont \textbf{\uline{Implementation:}}\par}\par

\begin{justify}
The proposed system will consist of a gesture-controlled glove/band connected to various controllable electrical devices such as lights or fans and even doors or windows. This will solve the issue of mobility to a great extent. The glove will also have a smart band component that keeps track of heart rate and pressure sensors to check if the user has fallen down. The current location of the user in the house will also be tracked using the gloves. This would be particularly useful\textbf{ }to keep track of the person and personalize the behaviour of electrical devices.
\end{justify}\par

\begin{justify}
An AI agent will monitor the activity cycle generated by the user daily and determine the next course of action. This AI agent will serve as a base for all other modules. The data will be stored on the cloud and an application to access and monitor the knowledge gained from this data will be developed. The main function of the application is to notify the caretaker and/or family members in case of emergencies, be it any injury or illness.
\end{justify}\par

\begin{justify}
The proposed system consists of the following modules:
\end{justify}\par



%%%%%%%%%%%%%%%%%%%% Figure/Image No: 1 starts here %%%%%%%%%%%%%%%%%%%%

\begin{figure}[H]
	\begin{Center}
		\includegraphics[width=4.57in,height=3.69in]{./media/image1.png}
	\end{Center}
\end{figure}


%%%%%%%%%%%%%%%%%%%% Figure/Image No: 1 Ends here %%%%%%%%%%%%%%%%%%%%

\par

\begin{enumerate}
	\item \textbf{Gesture Controlled Glove}\par

\begin{justify}
The glove consists of various sensors, which will be connected to the hub. The sensor will then map the readings of the gesture performed and accordingly perform the task. It will contain other sensors to detect abnormal heart rate, falls etc too.
\end{justify}\par

	\item \textbf{AI Agent for ADL and Activity Recognition }\par

\begin{justify}
An AI agent monitors the activity cycle generated by the user daily by using fuzzy logic and then determining the next course of action. The AI agent also helps users with Alzheimer's by keeping a check on their usual day to day activities and reporting anything out of the ordinary.
\end{justify}\par

\begin{justify}
Human activity recognition is concerned with identifying the specific movement or action of a person based on sensor data. We divide the input signal data into windows of signals, where a given window may have one to a few seconds of observation data. A given window of data may have multiple variables, such as the x, y, and z axes of an accelerometer sensor. One window would represent one sample. We extract informative features from each time window. For each time window, we extract a single feature vector f with acceleration vector of an axis X, Y, and Z, respectively as follows: 
\end{justify}\par



%%%%%%%%%%%%%%%%%%%% Figure/Image No: 2 starts here %%%%%%%%%%%%%%%%%%%%

\begin{figure}[H]
	\begin{Center}
		\includegraphics[width=5.89in,height=0.39in]{./media/image2.png}
	\end{Center}
\end{figure}


%%%%%%%%%%%%%%%%%%%% Figure/Image No: 2 Ends here %%%%%%%%%%%%%%%%%%%%

\par

\begin{justify}
The feature vector of the time window is used as the input to the classifier. Along with location coordinates, we can classify the activity as sitting, standing, sleeping etc.
\end{justify}\par


\vspace{\baselineskip}
	\item \textbf{Mobile App and Cloud Storage}\par

\begin{justify}
The data will be stored in a centralised database on the cloud, Firebase, which will provide data to the AI agent and application. The data will be in the form of raw sensor readings along with time stamps. It will also contain logs of the electrical appliance activities, i.e. when and what was switched on/off.
\end{justify}\par

\begin{justify}
An application will be built in Android to access and monitor the knowledge gained from this data will be developed.  It will send regular updates and alert notifications to the caretaker and/or family members in case of emergencies, be it any injury or illness.
\end{justify}\par


\vspace{\baselineskip}
	\item \textbf{A.I.D.E Hub}
\end{enumerate}\par

\begin{justify}
The AI agent is present in this centralised controller for the entire system. Any enhancement or modification to the raw data received from the cloud will be done at the hub. The hub will also have Temperature and Gas Sensors attached to it. All the controllable electrical devices will be connected via WiFi module to this central hub.
\end{justify}\par


\vspace{\baselineskip}


%%%%%%%%%%%%%%%%%%%% Figure/Image No: 3 starts here %%%%%%%%%%%%%%%%%%%%

\begin{figure}[H]
	\begin{Center}
		\includegraphics[width=5.01in,height=6.56in]{./media/image3.png}
	\end{Center}
\end{figure}


%%%%%%%%%%%%%%%%%%%% Figure/Image No: 3 Ends here %%%%%%%%%%%%%%%%%%%%

\par


\vspace{\baselineskip}
{\fontsize{14pt}{16.8pt}\selectfont \textbf{\uline{Feasibility:}}\par}\par

\begin{justify}
Currently, disabled-friendly facilities are lacking. They have issues of mobility, speech impairment or possibility of memory loss. Families are unable to give all their time to constantly see to the needs of such people. 
\end{justify}\par

\begin{justify}
A disability-friendly home automation system would have significant social benefits as it helps the disabled perform tasks that they find difficult or inconvenient. It will be extremely useful for those who wish to not avail of assisted living facilities. The system will provide similar facilities to provide some basic sense of independence to them. At the same time, this system will also help the caretakers and family members of those with special needs as a handy tool to remotely monitor their well-being which will reduce their tension. In case of emergencies such as rapid heart rate or falling, the system will detect it and elicit prompt action.
\end{justify}\par

\begin{justify}
For the elderly with memory loss or Alzheimer’s disease it will help by sending notification updates to the person himself /herself and their relatives to remind them to eat/drink etc. 
\end{justify}\par


\vspace{\baselineskip}
{\fontsize{14pt}{16.8pt}\selectfont \textbf{\uline{References:}}\par}\par

\begin{justify}
[1] Home Automation, Wikipedia <Link: \href{https://en.wikipedia.org/wiki/Home_automation}{https://en.wikipedia.org/wiki/Home\_automation}>
\end{justify}\par

\begin{justify}
[2] Disability in India Is Still All About the Able, The Diplomat [Online]
\end{justify}\par

\begin{justify}
<Link: \href{https://thediplomat.com/2018/08/disability-in-india-is-still-all-about-the-able}{https://thediplomat.com/2018/08/disability-in-india-is-still-all-about-the-able} [1st August 2018]>
\end{justify}\par

\begin{justify}
[3] Ravi, A., et al. "Smart Voice Recognition Based Home Automation System for Aging and Disabled People." International Journal of Advanced Scientific Research $\&$  Development (IJASRD) 5.1 (2018): 11-18.
\end{justify}\par

\begin{justify}
[4] Ansah, Albert Kwansah, Jeffrey Antwi Ansah, and Stephen Anokye. "Technology for the Aging Society-A Focus and Design of a Cost Effective Smart Home for the Aged and Disabled." \textit{Proceedings of the World Congress on Engineering and Computer Science}. Vol. 1. 2015.}
\end{justify}\par

\begin{justify}
[5] Kurti, Jasmin, and Ph D. GünayKarl$\iota$ . "Health care in home automation systems with speech recognition and mobile technology." \textit{American Journal of Engineering Research (AJER)} 3.3 (2014): 262-265.}
\end{justify}\par

\begin{justify}
[6] El-Basioni, Basma M. Mohammad, Sherine Mohamed Abd El-Kader, and Hussein S. Eissa. "Independent living for persons with disabilities and elderly people using smart home technology." \textit{International Journal of Application or Innovation in Engineering and Management} 3.4 (2014): 11-28.}
\end{justify}\par

[7] Venture Beat, Kyle Wiggers <Link:\href{https://venturebeat.com/2019/01/02/carepredict-raises-9-5-million-for-ai-wearable-that-monitors-seniors-health/}{https://venturebeat.com/2019/01/02/carepredict-raises-9-5-million-for-ai-wearable-that-monitors-seniors-health/} [2\textsuperscript{nd} January 2019]>\par


\printbibliography
\end{document}